\chapter*{Synopsis}\addcontentsline{toc}{section}{Synopsis}
The simplicial homology global optimisation (SHGO) algorithm is a general purpose global optimisation algorithm based on applications of simplicial integral homology and combinatorial topology. SHGO approximates the homology groups of a complex built on a hypersurface homeomorphic to a complex on the objective function. This provides both approximations of locally convex subdomains in the search space through Sperner's lemma (\citeauthor{Sperner1928}, \citeyear{Sperner1928}) and a useful visual tool for characterising and efficiently solving higher dimensional black and grey box optimisation problems. This complex is built up using sampling points within the feasible search space as vertices. The algorithm is specialised in finding all the local minima of an objective function with expensive function evaluations efficiently which is especially suitable to applications such as energy landscape exploration. SHGO was initially developed as an improvement on the topographical global optimisation (TGO) method first proposed by \citeauthor{Torn1986} (\citeyear{Torn1986, Torn1990, Torn1992}). It is proven that the SHGO algorithm will always outperform TGO on function evaluations if the objective function is Lipschitz smooth. In this dissertation SHGO is applied to non-convex problems with linear and box constraints with bounds placed on the variables. Numerical experiments on linearly constrained test problems show that SHGO gives competitive results compared to TGO and the recently developed Lc-DISIMPL algorithm (\citeauthor{Paul2016}, \citeyear{Paul2016}) as well as the PSwarm and DIRECT-L1 algorithms. Furthermore SHGO is compared with the TGO, basinhopping (BH) and differential evolution (DE) global optimisation algorithms over a large selection of black-box problems with bounds placed on the variables from the SciPy (\citeauthor*{scipy}, \citeyear{scipy}) benchmarking test suite. A Python implementation of the SHGO and TGO algorithms published under a MIT license can be found from \url{https://bitbucket.org/upiamcompthermo/shgo/}.
\bigskip

\noindent \textbf{Keywords:} Global optimisation, SHGO, Computational homology
\begin{flushleft}\bf{Mathematics Subject Classification (2010)} \normalfont 90C26 Nonconvex programming, global optimisation
\end{flushleft}

% Local Variables: 
% TeX-master: "thesis" 
% End:

